% !TeX program=xelatex
\documentclass[11pt]{article} % use larger type; default would be 10pt

\usepackage{fontspec}
  \setmonofont{FreeMono}

\usepackage{amsmath}
\usepackage{mathtools}

\title{E-Expressions}
% \date{}
\author{Okuno Zankoku}



\begin{document}
\maketitle

I'm \emph{so done} writing parsers for my language experiments.
I always want a whole host of ergonomic features, decent speed, and good error reporting, but it always takes \emph{so long} when what I want to be doing is playing with semantics.
So, I resolved to write one last parser which can be re-used across all my experiments.
This parser would deal with all of the character manipulation, leaving the question of concrete syntax to one of simple pattern-matching against an abstract data type.
Thus, I implemented ``ergonomic expressions'' (or maybe ``elaborate''? ``extended''?) or ``e-exprs''\footnote{I'll let you decide how to pronounce that \texttt{:)}, but I like ``\emph{eks}-per''} for short.

{
  Why not just s-expressions?
  Don't get me wrong, my father taught me Scheme last century; it was my first programming language, and to this day I always install Racket on new computers because I never know when I might want to play.
  Nevertheless---controversial take---I think Lisps have too many parenthesis,%
    \footnote{Also, Lisps use parenthesis in a strange way: they indicate function application but never grouping, whereas mathematical practice, and especially when working with the lambda calculus, is to use parentheses mostly for grouping.
    Of course, many languages, and even standard mathematics, use them for both grouping and function call, but Haskell and ML don't.
    }
    and that is a sign of a too low-level interface.
  A user-facing syntax is a user interface, and as such, it should have basic amenities.
  Lisp intensionally lacks ergonomic features because its users largely believe that s-expressions are so beneficial to macro-writing that it outweighs the cost of the harsh ergonomics.
  More recent languages show that any language that supports abstract data types can manipulate its own arbitrarily complex syntax as easily as a Lisp: the benefit is wholly imaginary, and therefore cannot be worth even a small cost.
}

Overall though, e-exprs are not so different from s-exprs---there are just more ways to combine them into larger expressions.
Let's quickly go over the distinctive features of e-exprs.

\section{A Whirlwind Tour}

Perhaps the largest conceptual distinction is that combination is not always function\footnote{or special form, or macro} application.
Parenthesis are conceptualized as grouping, though a specific language may interpret them differently.
To that end, there are two other simple means of grouping: square brackets and curly braces.
Unlike some proposed s-expr extensions, parens, brackets, and braces are kept distinct.

Ironically, s-exprs use a lot of line noise to build lists.
I therefore included the ability to group expressions by comma-separating them.
\begin{center}
\begin{verbatim}
'((f x) (g y) z)
\end{verbatim}
\begin{verbatim}
[f x, g y, z]
\end{verbatim}
\end{center}
Adding a leading or trailing comma doesn't change anything.
You can even have a comma-separated list with zero items \texttt{(,)} or one item \texttt{(a,)}/\texttt{(,a)}.
Semicolons work the same way, and add a second layer of grouping.
Colons are similar, but separate exactly two subexpressions from each other.
An ellipsis (spelled with two dots: \texttt{..}) may separate two expressions, but wither side may be empty.

Another way to combine expressions is to ``chain'' them together with dots, just as you would in almost any programming language for accessing a field of a record.
\begin{center}
\begin{verbatim}
(point-x (player-position (world-player world)))
\end{verbatim}
\begin{verbatim}
world.player.position.x
\end{verbatim}
\end{center}
Unlike other languages, you can chain any two e-exprs together with dots.
Additionally, the dot is optional if the next expression in the chain is surrounded by a bracketing form (parens, square brackets, curly braces, or an indented block, which we will soon see).
This means that syntaxes that look like field access, array indexing, or function application are represented by the same, unified syntax.
\begin{center}
\begin{verbatim}
matrix[2, 3]
point3d{x, y}.scale(4)
\end{verbatim}
\end{center}
Bear in mind that the lack of whitespace around the dot is critical: adding whitespace changes the meaning, as we will see later.

The most visually-apparent change is the addition of indentation-sensitivity.
Importantly, an indented block is always preceded by an end-of-line colon.
One of the most iconic Lisp notions, \texttt{cond} looks terrible, but applying indentation-sensitivity, chaining, and colon-separation can clean it up significantly.
\begin{center}
\begin{verbatim}
(cond
  ((p? a) (foo a))
  ((p? b) (foo b))
  (else foo c))
\end{verbatim}
\begin{verbatim}
cond:
  p? a: foo a
  p? b: foo b
  else: foo c
\end{verbatim}
\end{center}
An indented block can also appear on its own rather than as part of a chain.
Note the space before the colon that ensures the block isn't chained with the \texttt{z} argument.
\begin{center}
\begin{verbatim}
def big_function x y z :
  xy = combine x y
  z' = transpose z
  z' * xy
\end{verbatim}
\end{center}


Remember that there isn't much of a denotational semantics for e-exprs; they are pretty much exactly what you write.
In the case of the last example, we simply have a sequence of expressions, the last of which is a block containing three subexpressions, each of which is just a list of symbols.
Although it appears we are using binary operators, e-exprs do not have to support them.
A layer on top of e-exprs can be responsible for detecting and rewriting any infix---or indeed mixfix---operators, and indeed my e-exprs library supplies just such a system.

E-exprs also support a handful of other smaller features.
String templates \texttt{"Hello, `titlecase name`!"} are a feature I can no longer do without.
Unicode symbols \texttt{λ x: x + 1} allow for more math-like notation if desired.
Heredocs allow for large strings to be used as documentation, quick-and-dirty data blocs, or to embed other languages that a macro system can pick up and translate.
Synthetic infix functions \texttt{this .(not a real method) arg1 (arg 2)} allow for function call syntax to be re-ordered to express a more naturalistic reading.

E-exprs \emph{don't} support any language-specific concepts like cons-dot, tick-quotation, quasi-quote/unquote/unquote-splicing, or reader macros.
I point out the Lisp ideas specifically, since Lisp is often accused of having ``no syntax'', but e-exprs also don't have any concepts from more conventional languages.
Even \texttt{[1, 2, 3]} is not a list---as far as e-exprs are concerned, it is a comma-separated sequence of expressions enclosed in square brackets.
If you want that to be a list in your language then it can be, or it could be something completely different; in fact, as part of a chain, it might make sense if it were a multi-dimensional array access.

The point is that e-exprs don't solve the problem of concrete syntax.
Instead, e-exprs deal with the messy character stream to produce a straightforward tree data structure containing widespread notations.
Bridging the gap from an e-expr to an abstract syntax is much easier because such a tree is amenable to simple functional pattern matching.
I hope to be able to take any formal programming calculus from the literature and have a e-expr based concrete syntax built inside of fifteen minutes---one which I will enjoy writing in so that I can really experiment with the semantic ideas.

\section{E-expr Grammar}

In the following, we use grammatical notation drawn from a number of fields.
From regular expressions, we have
  parenthesis for (non-capturing) grouping,
  pipe $a \mid b$ for alternation,
  $r^?$ for optional elements,
  $r^\star$ and $r^+$ for repetition zero/one or more times respectively,
  and $r^{\{n\}}$ for $n$-times repetition.
A more advanced form is the difference of regular expressions $r_1 - r_2$ which matches all $r_1$ that \emph{don't} match $r_2$; regular expressions have been proven closed under subtraction, but it hasn't made it into major implementations.
Literal characters and strings are in \texttt{typewriter} font and surrounded by \texttt{'}single ticks\texttt{'}.
We use standard C-style character escapes for otherwise hard-to-typeset whitespace characters (space, tab, and newlines), as well as backslash itself; to be safe, we also backslash-escape single- and double-quotes and backtick.
We use character classes which can reference both literal characters and the names of common classes and those we define ourselves.
  We typeset the names of classes in \textit{italics}.
From BNF, we write 
  non-terminals in \textbf{boldface},
  but also allow the definition of character classes whose names are in \textit{italics}.

\subsection{Lexer}

\newcommand*\literal[1]{\texttt{'{#1}'}}
\newcommand*\notcharclass[1]{\mathopen{\lnot[\mathord:\;} #1 \mathclose{\;\mathord:]}}
\newcommand*\charclass[1]{\mathopen{[\mathord:\;} #1 \mathclose{\;\mathord:]}}

The first step in parsing an e-expression is to identify pre-tokens from a stream of UTF-8 characters.
The grammars for most pre-tokens mostly describe regular languages, but we have made some use of BNF notation to help keep the regular expressions small and understandable.
TODO: but... heredocs!
These pre-tokens may need to later be classified into tokens proper

\newcommand{\gis}{\mathrel{::=}}
\newcommand{\gor}{\mathrel{\phantom{\gis}\mathllap{\mid}}}

\begin{align*}
\intertext{These three bracketing forms are hardcoded as special, as are these five punctuation marks.
  Note that ellipsis \literal{..} is preferred over a single dot \literal{.}.}
  \mathbf{encloser} &\gis \charclass{\literal{()[]\{\}}} \\
  \mathbf{separator} &\gis \literal{,} \mid \literal{.} \mid \literal{..} \mid \literal{;} \mid \literal{:} \\
\intertext{Inline whitespace consists of newlines and tabs, but also}
  \mathbf{preSpace} &\gis \charclass{\literal{\textbackslash{ }\textbackslash t}}^+ \\
\intertext{a line-continuation is also treated as if it were inline whitespace.}
                    &\gor \literal{\textbackslash\textbackslash\textbackslash n} \\
\intertext{On the other hand, line-breaks may need to be distinguished from inline whitespace.}
  \mathbf{preNewline} &\gis \literal{\textbackslash n} \\
\intertext{Only line-comments are allowed; see the discussion below.}
  \mathbf{comment} &\gis \literal{\#} \; \notcharclass{\literal{\textbackslash n}}^+ \\
  symbolChar &= \charclass{letter, digit, mathSymbol, currencySymbol} \\
             &+ \charclass{\literal{\textasciitilde!@\$\%\textasciicircum\&*-\textunderscore=+|<>/?}} \\
             &- \charclass{\texttt{\textbackslash{ }\textbackslash t\textbackslash n\textbackslash r\texttt\#\textbackslash\textbackslash}} \\
             &- \charclass{\texttt{()[]{},.;:\textbackslash`\textbackslash'\textbackslash"}} \\
\intertext{TODO: I haven't decided if I really want to allow colons to start symbols.}
  \mathbf{symbol} &\gis (\charclass{symbolChar} - \charclass{\literal{+-}}^? \; \charclass{digit}) \; \charclass{symbolChar}^* \\
                  &\gor \literal{:} \charclass{symbolChar, \literal{:}}^+ \\
\intertext{FIXME: this doesn't admit non-decimal bases or scientific notation}
  \mathbf{number} &\gis \charclass{\literal{+-}} \; \charclass{digit}^+ \\
\intertext{Strings are enclosed in double-quotes, but an expression can be spliced into a string by enclosing the expression in backticks.
  At this stage in parsing, I am not interested in matching string fragments up and scooping up the spliced expressions in-between to form the entire template: this is just the elxer for goodness' sake!
  In Any case, i means that strings might start and end with any mix of double-quote and backtick.}
  \mathbf{stringPart} &\gis \charclass{\literal", \literal`}\;\mathbf{stringContent}^\star\;\charclass{\literal{\textbackslash"\textbackslash`}} \\
\intertext{Most UTF-8 characters are available for use directly in strings, with some obvious exceptions.
  Notably, strings are not allowed to cross line-boundaries without special syntax, since that's a common cause of syntax error.}
  \mathbf{stringContent} &\gis \notcharclass{\literal{\textbackslash"\textbackslash`\textbackslash r\textbackslash n\textbackslash\textbackslash}} \\
\intertext{The usual C-style escapes are available, as well as an escapse for backtick.
  We also include a \texttt{\textbackslash e} for ASCII 27 ``escape'', since it finds a little use in terminal interface programming.
  Also, we include \texttt{\textbackslash\&} which stands for the empty string, just in case two adjacent portions of string would be interpreted as an escape.}
                         &\gor \literal{\textbackslash\textbackslash} \; \charclass{\literal{\textbackslash\textbackslash0abefnrtv\textbackslash'\textbackslash"\textbackslash`\&}} \\
\intertext{The \texttt{\textbackslash x}, \texttt{\textbackslash u}, and \texttt{\textbackslash U} escapes allow for encoding arbitrary unicode codepoints into a string.
  While \texttt{\textbackslash U} is the most general, the others can be used when encoding bytes in a bytestring, or when only the Basic Multilingual Plane is required.
  The \texttt{\textbackslash U} form can be trickeier to use effectively: since it accepts any number of digits afterwards, a null-escape may need to be inserted to end the escape when it ocuurs before a $hexDigit$, e.g.\ \texttt{"\textbackslash U101234\textbackslash\&5"}.
  It accepts any number of characters because then it will be future-proofed in the unlikely event of Unicode (or a supersceding standards body) extending its codepoints beyond \texttt{0x10FFFF}.
  }
                         &\gor \literal{\textbackslash\textbackslash x} \; \charclass{hexDigit}^{\{2\}} \\
                         &\gor \literal{\textbackslash\textbackslash u} \; \charclass{hexDigit}^{\{4\}} \\
                         &\gor \literal{\textbackslash\textbackslash U} \; \charclass{hexDigit}^+ \\
\intertext{Finally, multi-line strings may explicitly be broken up across lines by enclosing the newline between backslashes.
  It is recommended to maintain the visual indicators of nesting by indenting the following line appropriately.
  Note that when this form is used, a newline character is \emph{not} inserted.}
                         &\gor \literal{\textbackslash\textbackslash} \; \literal{\textbackslash n} \; \charclass{\literal{\textbackslash{ }\textbackslash t}}^\star \literal{\textbackslash\textbackslash} \\
  hexDigit &= \charclass{\literal{0123456789ABCDEFabcdef}} \\
\intertext{Heredocs offer an alternative for embedding long strings into e-exprs, at the cost of not having string templating.
  They can be multi-line and do not require escape sequences anywhere, but they also do not allow escape sequences or templating.
  }
  \mathbf{heredoc}
    &\gis \literal{\textbackslash"\textbackslash"\textbackslash"} \; (\charclass{alphaNum}^*)^{=delim}\; \literal{\textbackslash n} \; \\
    &\qquad \mathbf{heredocLine}(delim)^* \\
    &\qquad \mathbf{heredocEnd}(delim) \\
\intertext{Although the (relatively) large number of supporting rules makes heredocs appear complex, they are essentially just triple-quoted strings, where a delimiter may be given just in case one of the lines must also start with a triple-quote.
  The $\mathbf{heredocLine}(delim)$ rule simply identify lines that don't end with $delim$ followed by triple quotes.
  }
  \mathbf{heredocLine}(delim)
    &\gis (\mathbf{line} - \mathbf{heredocEnd}(delim) \; \mathbf{line}) \; \literal{\textbackslash n} \\
  \mathbf{line}
    &\gis \notcharclass{\literal{\textbackslash n}}^* \\
  \mathbf{heredocEnd}(delim) &\gis delim \; \literal{\textbackslash"\textbackslash"\textbackslash"} \\
TODO: char \\
TODO: rational
\end{align*}

I've chosen to not have block comments.
Block comments are fragile unless they are able to nest inside each other.
For historical reasons, most languages do not have nesting block comments, so I've found that text editors do not bother implementing good support for nesting block comments.
The last thing I want is to ask a potential e-expr user to change their workflow, so I've simply ignored block comments altogether.
Instead, your text editor likely has good support for commenting out all selected lines at once, which is all a block comment does.
Besides, using only line comments makes changes more easily spotted when using line-based version control.

The set of symbol characters is \emph{not} wisely chosen overall.
I just took a peek at what Unicode had to offer, picked some reasonable-looking properties.
Outside of the ASCII portion and some well-known mathematical symbols, I am open to critique, especially since I have done none myself.
I hope for now that this is a reasonable enough set so as not to cause confusion, but I expect it will be important to reduce the confusion caused by similar-looking glyphs (e.g. at the moment capital A and capital alpha look the same, but will be treated differently in code).
Nevertheless, you can see that I've tried to include as many symbols as possible that are available on US/UK keyboards (programming seems to have settled on this as a de-facto standard, and I don't have the knowledge to change it; if you have gripes about what characters on your local keyboards you would most like to use/avoid in code, please get in touch!)
I have explicitly ruled out whitespace and comment-related characters (for obvious reasons), as well as a number of characters used as punctuation in e-exprs.


I've only allowed the one form of line-break, because who wants to deal with OS-specific line-break sequences in 2021?
Perhaps your text editor can be configured, and perhaps your VCS is ``smart'', but why not agree to stop paying the complexity cost when all we get out of it is that one company is pampered?
That said, as little as single open issue might change my mind.


FIXME: shouldn't I allow $\mathbf{symbol} ::= \literal{\textbackslash\textbackslash}\;\charclass{symbolChar}^+$?

\section{Token Stream Cleanup}

After the lexer identifies tokens in the input Unicode stream, we are left with a stream of pre-tokens which must now be validated and normalized.

TODO

\subsection{Parser}

\section{E-exprs as a System}

TODO: location tracking, error recovery, the pattern-matching library, the various output formats, the mixfix algorithm

\end{document}
